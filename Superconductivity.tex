\documentclass[prb,preprint]{revtex4-1} 
% The line above defines the type of LaTeX document.
% Note that AJP uses the same style as Phys. Rev. B (prb).

\usepackage{amsmath}  % needed for \tfrac, \bmatrix, etc.
\usepackage{amsfonts} % needed for bold Greek, Fraktur, and blackboard bold
\usepackage{graphicx} % needed for figures
\usepackage{tabularx}

\begin{document}

\title{Superconductivity}
% In a long title you can use \\ to force a line break at a certain location.

\author{Jiajun Shi}
\email{jshi15@amherst.edu} 
\affiliation{Department of Physics, Smith College, Northampton, MA 01063}
\author{He Claudia Yun}
\email{hyun@smith.edu}
\affiliation{Department of Physics, Smith College, Northampton, MA 01063}

% See the REVTeX documentation for more examples of author and affiliation lists.

\date{\today}

%____________abstract____________________________________________

\begin{abstract}

\end{abstract}

\maketitle 

%____________Introduction____________________________________________
\section{Introduction}

%____________Experiment____________________________________________
\section{Experiment}
Temperature and resistance are measured for two different materials: $YBa_{2}Cu_{3}O_{7}$ (YBCO) and $Bi_{2}Sr{2}Ca_{2}Cu_{3}O_{9}$ (BSCCO). The experiments are done in two different ways for different materials.\\
\subsection{YBCO}
The setup of YBCO experiment is shown in the Fig. \ref{setup}. The device (Fig. \ref{sample}), including the YBCO sample, a thermocouple, a current probe and a voltage probe, is put in a plastic-foam cup and immersed in sands, which serve as insulator. The box variable resistor in Fig. \ref{setup} is only used for keeping the free wires at a fixed place. \\

\begin{figure}[h]
\centering
\includegraphics[width=14cm]{ybcosetup.jpg}
\caption{Picture of the experiment setup}
\label{setup}
\end{figure}

\begin{figure}[h]
\centering
\includegraphics[width=14cm]{ybcosample.jpg}
\caption{Picture of the device, including YBCO sample (the black disk on top), thermocouple (brown lead), current (black leads) and voltage probes (yellow leads)}
\label{sample}
\end{figure}

When superconducting, the material has a very low resistance, which means the resistance of the leads of the probes becomes non-negligible. Therefore a method called four point probe is employed so that only the resistance of the material is measured, not that of the leads. The circuit is shown in Fig. \ref{fpp}. The black rectangle at the bottom represents the YBCO sample, an ammeter is connected in series with the sample through leads 1 and 4, and a voltmeter is connected in parallel with the sample through leads 2 and 3. Then the resistance of the sample could be calculated using Ohm's Law:

\begin{equation}
R=\frac{V}{I}
\label{ohm}
\end{equation}

where V and I are readings of the voltmeter and ammeter respectively. The key point here is that leads 2 and 3 have to be inside leads 1 and 4, in terms of position on the sample. Picture the meters as ideal meters plus lead resistance. The advantage of doing so is that the ammeter approximately measures the current through the sample, because the resistance of the voltmeter is so much bigger than that of the superconductor that the current through the voltmeter is very close to zero; and the voltmeter only measures the voltage across the sample. However, i f the position of the ammeter and voltmeter were switched, as the second circuit shown in Fig. \ref{fpp}, the voltmeter would measure the voltage across the lead resistance of the ammeter and the sample resistance connected in parallel. The total resistance of the two in parallel would differ from the sample resistance a lot because when superconducting, the resistance of the sample is comparable to the resistance of the lead. \\

\begin{figure}[h]
\centering
\includegraphics[width=8cm]{fourpointprobe2.png}
\includegraphics[width=8cm]{fourpointprobe3.png}
\caption{Schematic of four point probe. The right is the bad configuration. The circuits are drawn with online tool Scheme-it \cite{drawcircuit}.}
\label{fpp}
\end{figure}

The sample is connected to two digital multimeters (DMM) (top two in Fig. \ref{meters}) and one power source {bottom one in Fig. \ref{meters}}. The top one is measuring the voltage across the thermocouple, and the voltage can be translated into temperature according to the table in Fig. \ref{temp}. \\

\begin{figure}[h]
\centering
\includegraphics[width=10cm]{ybcometers.jpg}
\caption{The DMMs and power supply used. The top one measures the voltage of the thermocouple, the middle one measures the voltage across the sample, but the value is not shown on the screen under delta mode. And the bottom one is a power source that is in delta mode, and it reads the resistance of the YBCO sample.}
\label{meters}
\end{figure}

\begin{figure}[h]
\centering
\includegraphics[width=14cm]{temperature.pdf}
\caption{The correlation between the temperature and the voltage across the thermocouple}
\label{temp}
\end{figure}

The source is the "DC power supply" in Fig. \ref{fpp} and Fig. \ref{fpp2}. Although it is labelled as DC power source, it is actually in delta mode, which means it is outputting a current alternating between positive and negative. The advantage of delta mode is that it can eliminate the affect of offsets measured by the voltmeter. When a positive current is passing through the sample, the voltmeter measures an offset and the real voltage signal across the sample, $V_{+}=V_{offset}+V_{signal}$; if the current is reversed, the voltage across the sample would also flip sign, but the offset would remain the same, $V_{-}=V_{offset}-V_{signal}$. From $V_{+}$ and $V_{-}$, the voltage due to the sample can be calculated. The middle DMM reads the two voltages and calculates $V_{signal}$ automatically. The middle DMM is also connected to the power supply at the back. The power source reads $V_{signal}$ from the middle DMM and then divides it by the current to get the resistance of the superconductor, which is shown on the screen in unit of m$\Omega$. By observing the temperature dependence of the resistance, the critical temperature of the superconductor could be found out. \\

However, the drawback of this setup is that we could not make computer record the data. So we videotaped the DMMs while the sample was cooled down by liquid nitrogen, and then transcribed the thermocouple voltage and superconductor resistance into analysis tool, i.e. Igor Pro, by hand. To test the current dependence of the critical temperature of the material, four sets of data are collected, with current output of the power supply being 0.1mA, 1mA, 10mA and 100mA respectively.\\

\subsection{BSCCO}

The setup of the experiment on BSCCO is shown in Fig. \ref{bsccosetup}. The device is also put in a plastic-foam cup and covered by sands. This device is designed in the same manner as the YBCO sample using four point probe.

\begin{figure}[h]
\centering
\includegraphics[width=14cm]{bsccosetup.jpg}
\caption{The setup of experiment of BSCCO}
\label{bsccosetup}
\end{figure}

The thermal couple is connected to the left lock-in amplifier in Fig. \ref{bsccosetup}, which has a time constant of 1s and a gain of 1000. The output is fed to a DMM which is read by the computer. \\

A Model SR830 DSP lock-in amplifier (simplified as SR830 lock-in amplifier, gray equipment on the lower right in Fig. \ref{bsccosetup}) is employed in this experiment. The most important feature of this lock-in amplifier is that it feeds the sample a sine signal and then it detects the in-phase and out-of-phase feedbacks from the sample separately, with in-phase feedback shown in channel 1 (left) and the out-of-phase one shown in channel 2 (right). \\

When wires are close together, they form a capacitor. When they curl, they become inductors. In both situations the wires bring in noise if we only want to know the voltage across the resistor. Normally the noise would be so small that it could be safely neglected. However, the resistance of the superconductor drops to such a low level that that noise becomes dominant. To distinguish between signal and noise, a sine input is introduced. The resistor outputs a voltage that is in-phase with the input, while the "capacitors" and "inductors" formed by wires yield voltage that has a $\pi/2$ phase delay. We could filter out the noise by only taking the in-phase voltage. \\

SR830 lock-in amplifier has a sine voltage output, whose peak to peak value is measure to be $2.80 \pm 0.02 V$. Therefore it has the root mean square value is $0.990\pm0.007V$. This output is fed to the input of a breakout box (the black box in the upper right in Fig. \ref{bsccosetup}), and then converted to a current through a resistor of $1k\Omega$. This current now acts as the "DC power supply" in Fig. \ref{fpp}, and is introduced into the circuit through the black wires (the current probe). The yellow leads (the voltage probe) are connected to a transformer preamplifier with a gain of 500, which is then connected back to the SR830 lock-in amplifier. The in-phase channel (CH1) is read by another DMM which ultimately communicates with the computer. LabView is used to collect the readings from the two DMMs: the voltage from the thermal couple and the in-phase voltage signal from the SR830 lock-in amplifier. \\

The resistance of the superconductor can be calculated by

\begin{equation}
R_{superconductor} = \frac{V_{in-phse}/g}{I}
\label{rofs}
\end{equation}

where

\begin{equation}
I=\frac{V_{rms}}{R_{breakout}}
\label{current}
\end{equation}

and $V_{in-phse}$ is the in-phase voltage measured, $g=500$ is the gain of the transformer, $V_{rms}=0.990\pm0.007V$ is the maximal input sine voltage from SR830 lock-in amplifier and $R_{breakout}=1k\Omega$ is the resistance in the breakout box. And the temperature of the sample could be converted from the voltage across the thermal couple using the table in Fig. \ref{temp}.\\


%____________Results____________________________________________
\section{Results}

\subsection{YBCO}
The temperature, converted from thermal couple voltage and resistance of the superconductor are collected and plotted, as shown in Fig, \ref{ybco10ma}. Then the high temperature part, the transit part and the low temperature are fitted with lines. The intersections of the high temperature and the transit part, and of the low temperature and the transit part are found. The critical temperature is then calculated by averaging the temperatures of the two intersecting points. Fig. \ref{ybcoanalysis} shows the analysis. (Figures of analyses of other data sets are shown in the appendix.)

\begin{figure}[h]
\centering
\includegraphics[width=14cm]{ybco10ma_raw.png}
\caption{Resistance vs. temperature of YBCO when current is 10mA}
\label{ybco10ma}
\end{figure}

\begin{figure}[h]
\centering
\includegraphics[width=14cm]{ybco10ma.png}
\caption{Analysis of data. The percentage uncertainties are indicated in the parenthesis. }
\label{ybcoanalysis}
\end{figure}

The critical temperatures of YBCO at different currents are found and listed in Table. \ref{ybcodata}. The critical temperature in the last row are calculated from data collected by using the BSCCO setup for comparison. \\

It can be seen that the uncertainties for the critical temperatures are huge, which is a result of the small number of data points we have. And that is because we have to transcribe the data from videos into computers by hand, which is not a very efficient or economical method.\\

\begin{table}[h]
\centering
\caption{YBCO Critical Temperature}
\begin{ruledtabular}
\begin{tabular}{ c c c}
Current & Critical Temperature & Uncertainty\\
(mA) & (K) & (\%)\\
\hline
0.1 & 115.9 & 60 \\
1 & 117.6 & 73 \\
10 & 119.4 & 23 \\
100 & 115.5 & 53 \\
N/A & 109.5 & 3.8 \\
\end{tabular}
\end{ruledtabular}
\label{ybcodata}
\end{table}

\subsection{BSCCO}
Only one set of data is obtained for BSCCO. The raw data is shown in Fig. \ref{bsccoraw}. The same line-fitting is done to the BSCCO data. However, the data of BSCCO are not as clean as that of YBCO: the high temperature part (low thermal couple voltage part) does not show an obvious linear relation. Therefore only the first linear section before the transit part is fitted (thermal couple voltage ranges from around 3.5V to 4.5V). The analysis is shown in Fig. \ref{bsccoanalysis}.\\

\begin{figure}[h]
\centering
\includegraphics[width=14cm]{bscco_heating_raw.png}
\caption{Raw data of BSCCO. The x axis is the thermal couple voltage, which can be converted to temperature. The y axis is the in-phase signal voltage, which can be converted to resistance using Eq. \ref{rofs}.}
\label{bsccoraw}
\end{figure}

\begin{figure}[h]
\centering
\includegraphics[width=14cm]{bscco_heating.png}
\caption{Analysis of BSCCO data}
\label{bsccoraw}
\end{figure}



%____________Discussion____________________________________________
\section{Discussion}

%____________Conclusion____________________________________________
\section{Conclusion}



\begin{acknowledgments}

We gratefully acknowledge Nathanael Fortune and Dana Parsons, who helped with the experimentation and editing of this experiment. This work was supported by the Smith College Physics Department.

\end{acknowledgments}


\begin{thebibliography}{99}

%\bibitem{wik} Double Slit Experiment, \url{<http://upload.wikimedia.org/wikipedia/commons/c/c2/Single_slit_and_double_slit2.jpg/>}.
\bibitem{drawcircuit} Scheme-it, \url{http://www.digikey.com/schemeit}
\bibitem{energy} Optical Pumping, \url{http://internal.physics.uwa.edu.au/~stamps/2006Y3Lab/SteveAndBlake/theoretical.html}

\end{thebibliography}

%\newpage   % Start a new page for tables

%____________Appendix_____________________________________________
\clearpage
\begin{appendix}
\section{Figures of analyses}

\begin{figure}[h]
\centering
\includegraphics[width=11cm]{ybco01ma.png}
\caption{Analysis of data at current equal to 0.1mA }
\label{ybcoanalysis2}
\end{figure}

\begin{figure}[h]
\centering
\includegraphics[width=11cm]{ybco1ma.png}
\caption{Analysis of data at current equal to 1mA }
\label{ybcoanalysis3}
\end{figure}

\begin{figure}[h]
\centering
\includegraphics[width=11cm]{ybco100ma.png}
\caption{Analysis of data at current equal to 100mA}
\label{ybcoanalysis4}
\end{figure}

\end{appendix}

\end{document}
