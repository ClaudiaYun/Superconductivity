\documentclass[prb,preprint]{revtex4-1} 
% The line above defines the type of LaTeX document.
% Note that AJP uses the same style as Phys. Rev. B (prb).

\usepackage{amsmath}  % needed for \tfrac, \bmatrix, etc.
\usepackage{amsfonts} % needed for bold Greek, Fraktur, and blackboard bold
\usepackage{graphicx} % needed for figures
\usepackage{tabularx}

\begin{document}

\title{Superconductivity}
% In a long title you can use \\ to force a line break at a certain location.

\author{Jiajun Shi}
\email{jshi15@amherst.edu} 
\affiliation{Department of Physics, Smith College, Northampton, MA 01063}
\author{He Claudia Yun}
\email{hyun@smith.edu}
\affiliation{Department of Physics, Smith College, Northampton, MA 01063}

% See the REVTeX documentation for more examples of author and affiliation lists.

\date{\today}

%____________abstract____________________________________________

\begin{abstract}

\end{abstract}

\maketitle 

%____________Introduction____________________________________________
\section{Introduction}

%____________Experiment____________________________________________
\section{Experiment}
Temperature and resistance are measured for two different materials: $YBa_{2}Cu_{3}O_{7}$ (YBCO) and $Bi_{2}Sr{2}Ca_{2}Cu_{3}O_{9}$ (BISCO). The experiments are done in two different ways for different materials.\\
\subsection{YBCO}
The setup of YBCO experiment is shown in the Fig. \ref{setup}. The device (Fig. \ref{sample}), including the YBCO sample, a thermocouple, a current probe and a voltage probe, is put in a plastic-foam cup and immersed in sands, which serve as insulator. \\

\begin{figure}[h]
\centering
\includegraphics[width=16cm]{ybcosetup.jpg}
\caption{Picture of the experiment setup}
\label{setup}
\end{figure}

\begin{figure}[h]
\centering
\includegraphics[width=16cm]{ybcosample.jpg}
\caption{Picture of the device, including YBCO sample (the black disk on top), thermocouple (brown lead), current (black leads) and voltage probes (yellow leads)}
\label{sample}
\end{figure}

When superconducting, the material has a very low resistance, which means the resistance of the leads of the probes becomes non-negligible. Therefore a method called four point probe is employed so that only the resistance of the material is measured, not that of the leads. The circuit is shown in Fig. \ref{fpp}. The black rectangle at the bottom represents the YBCO sample, an ammeter is connected in series with the sample through leads 1 and 4, and a voltmeter is connected in parallel with the sample through leads 2 and 3. Then the resistance of the sample could be calculated using Ohm's Law:

\begin{equation}
R=\frac{V}{I}
\label{ohm}
\end{equation}

where V and I are readings of the voltmeter and ammeter respectively. The key point here is that leads 2 and 3 have to be inside leads 1 and 4, in terms of position on the sample. Picture the meters as ideal meters plus lead resistance. The advantage of doing so is that the ammeter approximately measures the current through the sample, because the resistance of the voltmeter is so much bigger than that of the superconductor that the current through the voltmeter is very close to zero; and the voltmeter only measures the voltage across the sample. However, i f the position of the ammeter and voltmeter were switched, as shown in Fig. \ref{fpp2}, the voltmeter would measure the voltage across the lead resistance of the ammeter and the sample resistance connected in parallel. The total resistance of the two in parallel would differ from the sample resistance a lot because when superconducting, the resistance of the sample is comparable to the resistance of the lead. \\

\begin{figure}[h]
\centering
\includegraphics[width=8cm]{fourpointprobe2.png}
\caption{Schematic of four point probe}
\label{fp}p
\end{figure}

\begin{figure}[h]
\centering
\includegraphics[width=8cm]{fourpointprobe3.png}
\caption{Schematic of the other (bad) configuration of four point probe}
\label{fpp}
\end{figure}

The sample is connected to two digital multimeters (DMM) (top two in Fig. \ref{meters}) and one DC power source {bottom one in Fig. \ref{meters}}. 

\begin{figure}[h]
\centering
\includegraphics[width=12cm]{ybcometers.jpg}
\caption{The DMMs and power supply used. The top one measures the voltage of the thermocouple, the middle one measures the voltage across the sample, but the value is not shown on the screen under delta mode. And the bottom one is a power source that is in delta mode, and it reads the resistance of the YBCO sample.}
\label{meters}
\end{figure}

%\begin{figure}[h]
%\centering
%\includegraphics[width=16cm]{connection.jpg}
%\caption{Schematic of how the device is connected to the power supply and the meters}
%\label{connection}
%\end{figure}




\subsection{BISCO}


%____________Results____________________________________________
\section{Results}



%____________Discussion____________________________________________
\section{Discussion}

%____________Conclusion____________________________________________
\section{Conclusion}


\begin{acknowledgments}

We gratefully acknowledge Nathanael Fortune and Dana Parsons, who helped with the experimentation and editing of this experiment.  This work was supported by the Smith College Physics Department.

\end{acknowledgments}


\begin{thebibliography}{99}

%\bibitem{wik} Double Slit Experiment, \url{<http://upload.wikimedia.org/wikipedia/commons/c/c2/Single_slit_and_double_slit2.jpg/>}.
\bibitem{drawcircuit} Scheme-it, http://www.digikey.com/schemeit
\bibitem{energy} Optical Pumping, http://internal.physics.uwa.edu.au/~stamps/2006Y3Lab/SteveAndBlake/theoretical.html

\end{thebibliography}

%\newpage   % Start a new page for tables

\end{document}
